\documentclass[a4paper,twoside]{article}
\usepackage[T1]{fontenc}
\usepackage[bahasa]{babel}
\usepackage{graphicx}
\usepackage{graphics}
\usepackage{float}
\usepackage[cm]{fullpage}
\pagestyle{myheadings}
\usepackage{etoolbox}
\usepackage{setspace} 
\usepackage{lipsum} 
\setlength{\headsep}{30pt}
\usepackage[inner=2cm,outer=2.5cm,top=2.5cm,bottom=2cm]{geometry} %margin
% \pagestyle{empty}

\makeatletter
\renewcommand{\@maketitle} {\begin{center} {\LARGE \textbf{ \textsc{\@title}} \par} \bigskip {\large \textbf{\textsc{\@author}} }\end{center} }
\renewcommand{\thispagestyle}[1]{}
\markright{\textbf{\textsc{AIF401/AIF402 \textemdash Rencana Kerja Skripsi \textemdash Sem. Ganjil 2016/2017}}}

\onehalfspacing
 
\begin{document}

\title{\@judultopik}
\author{\nama \textendash \@npm} 

%tulis nama dan NPM anda di sini:
\newcommand{\nama}{Ega Prianto}
\newcommand{\@npm}{2013730047}
\newcommand{\@judultopik}{Deteksi Gerakan Kepala dengan Google Cardboard} % Judul/topik anda
\newcommand{\jumpemb}{1} % Jumlah pembimbing, 1 atau 2
\newcommand{\tanggal}{02/08/2016}
\maketitle

\pagenumbering{arabic}

\section{Deskripsi}
\textit{Virtual Reality} adalah teknologi yang mampu membuat penggunanya dapat berinteraksi dengan lingkungan buatan oleh komputer, suatu lingkungan yang sebenarnya ditiru atau hanya ada di dalam imajinasi. \textit{Virtual Reality} membuat pengalaman sensorik, di antaranya penglihatan, pendengaran, perabaan, dan penciuman. Gawai \textit{Virtual Reality} terbaru sekarang yaitu dengan menggunakan \textit{head-mounted display}, Google Cardboard salah satunya. \textit{Head-mounted display} adalah menempatkan layar di kepala, sehingga pengguna hanya dapat melihat tampilan yang ditampilkan oleh layar.

Google Cardboard adalah gawai murah yang terbuat dari kardus untuk dapat merasakan pengalaman \textit{virtual reality} dengan Android atau iOS \textit{smartphone}. Kita dapat membuat Google Cardboard kita sendiri secara gratis dengan mengunduh templatenya di website Google Cardboard. Template tersebut membantu dalam merakit kardus dengan dibentuk, dilipat dan digunting sedemikian rupa sehingga berbentuk kacamata. Bahan-bahan untuk merakit Google Cardboard hanyalah kardus, lem, dan lensa dengan spesifikasi tertentu.

Pada Gawai Google Cardboard cara pengguna memberikan \textit{input} kepada program sangatlah terbatas. Cara tersebut hanyalah dengan gerakan kepala dan tombol magnet. Tombol magnet ini pun terkadang tidak berfungsi dengan baik, karena bergantung pada medan magnet yang di deteksi oleh \textit{smartphone} yang digunakan. Cara lainnya agar dapat memberikan \textit{input} kepada program adalah dengan menghubungkan \textit{smartphone} yang digunakan dengan \textit{bluetooth controller}. 
Skripsi ini akan membuat aplikasi untuk menambahkan cara baru  memberikan \textit{input} pada Google Cardboard. Pada skripsi ini, akan dibuat dua buah perangkat lunak. Perangkat lunak pertama akan digunakan untuk menganalisis data yang didapat dari sensor-sensor pada Android. Perangkat lunak kedua akan dapat mendeteksi gerakan kepala penggunanya ketika sedang menggeleng atau mengangguk. Pada perangkat lunak kedua ini akan memberikan \textit{input} baru kepada program virtual reality. Jenis \textit{input} yang diberikan kepada komputer hanya ya(mengangguk) atau tidak(menggeleng).

Agar \textit{Virtual Reality} menggunakan Google Cardboard dapat berjalan dengan sempurna, dibutuhkan \textit{smartphone} yang memiliki 3 jenis sensor. Ketiga sensor itu adalah Magnetometer, Accelerometer, dan Gyroscope. Jika salah satu sensor itu tidak ada, tampilan gambar pada \textit{Virtual Reality} akan tidak akurat atau lambat. Magnetometer digunakan untuk mengetahui arah pandang pengguna. Accelerometer digunakan untuk mengetahui arah gaya gravitasi. Dan Gyroscope digunakan untuk mengetahui percepatan perputaran sudut kepala pengguna. Ketiga sensor ini juga harus menggunakan sensor 3 sumbu. Ketiga sensor tersebut tidak hanya berfungsi agar dapat menjalankan \textit{Virtual Reality} dengan Google Cardboard dan \textit{smartphone}, tetapi juga dapat berfungsi sebagai pendeteksi gerakan kepala.

\section{Rumusan Masalah}

\begin{itemize}
	\item Bagaimana cara menampilkan grafik data yang diambil dari sensor-sensor pada \textit{smartphone}?
	\item Bagaimana cara mendeteksi gerakan kepala dari data yang didapat dari sensor-sensor pada \textit{smartphone}?
	\item Bagaimana cara mendeteksi gerakan kepala dengan bantuan Google VR API?
\end{itemize}

\section{Tujuan}

\begin{itemize}
	\item Mengetahui cara untuk menampilkan grafik data dari sensor-sensor pada \textit{smartphone}.
	\item Mengetahui cara mendeteksi gerakan kepala dari data yang didapat dari sensor-sensor pada \textit{smartphone}.
	\item Mengetahui cara mendeteksi gerakan kepala dengan bantuan Google VR API?
\end{itemize}
\section{Deskripsi Perangkat Lunak}
Perangkat lunak pertama yang dibuat memiliki fitur minimal sebagai berikut :
\begin{itemize}
\item	Menampilkan grafik data untuk setiap sensor pada \textit{smartphone}.
\item	Merekam data untuk setiap sensor dalam rentang waktu menggeleng atau mengangguk.
\end{itemize}
Perangkat lunak kedua yang dibuat memiliki fitur minimal sebagai berikut :
\begin{itemize}
\item	Pengguna dapat memberikan masukkan kepada komputer dengan mengangguk atau menggeleng.
\end{itemize}

\section{Detail Pengerjaan Skripsi}

Bagian-bagian pengerjaan skripsi ini adalah sebagai berikut :
	\begin{enumerate}
		\item Melakukan studi literatur tentang Android SDK, Google VR SDK, dan Quaternion.
		\item Merancang dan membuat aplikasi untuk menampilkan grafik sensor-sensor pada \textit{smartphone} Android.
		\item Menganalisis aplikasi-aplikasi sejenis.
		\item Merekam dan menganalisis grafik dari sensor-sensor pada smartphone ketika mengangguk dan menggeleng.
		\item Menganalisis metode pendeteksi gerakan kepala.
		\item Merancang aplikasi untuk mendeteksi gerakan kepala
		\item Mengimplementasi algoritma pendeteksi gerakan kepala ke aplikasi virtual reality.
		\item Melakukan pengujian terhadap fitur-fitur yang sudah dibuat.
		\item Menulis dokumen skripsi.

	\end{enumerate}
\section{Rencana Kerja}
Tuliskan rencana anda untuk menyelesaikan skripsi. Rencana kerja dibagi menjadi dua bagian yaitu yang akan dilakukan pada saat mengambil kuliah AIF401 Skripsi 1 dan pada saat mengambil kuliah AIF402 Skripsi 2. Perhatikan contoh berikut ini :


\begin{center}
  \begin{tabular}{ | c | c | c | c | l |}
    \hline
    1*  & 2*(\%) & 3*(\%) & 4*(\%) &5*\\ \hline \hline
    1   & 5  & 5  &  &  \\ \hline
    2   & 5 & 5  &   & \\ \hline
    3   & 10  & 10  &  & \\ \hline
    4   & 10  & 10  &  & \\ \hline
    5   & 10  & 10  &  & \\ \hline
    6   & 10 &   & 10  & \\ \hline
    7   & 20  &   & 20 & \\ \hline
    8   & 10  &   &  10  & \\ \hline
    9   & 20  & 10  & 10  & {\footnotesize menulis dokumen skripsi hingga bab 3 pada S1}\\ \hline
    Total  & 100  & 50  & 50 &  \\ \hline
                          \end{tabular}
\end{center}

Keterangan (*)\\
1 : Bagian pengerjaan Skripsi (nomor disesuaikan dengan detail pengerjaan di bagian 5)\\
2 : Persentase total \\
3 : Persentase yang akan diselesaikan di Skripsi 1 \\
4 : Persentase yang akan diselesaikan di Skripsi 2 \\
5 : Penjelasan singkat apa yang dilakukan di S1 (Skripsi 1) atau S2 (skripsi 2)

\vspace{1cm}
\centering Bandung, \tanggal\\
\vspace{2cm} \nama \\ 
\vspace{1cm}

Menyetujui, \\
\ifdefstring{\jumpemb}{2}{
\vspace{1.5cm}
\begin{centering} Menyetujui,\\ \end{centering} \vspace{0.75cm}
\begin{minipage}[b]{0.45\linewidth}
% \centering Bandung, \makebox[0.5cm]{\hrulefill}/\makebox[0.5cm]{\hrulefill}/2013 \\
\vspace{2cm} Nama: \makebox[3cm]{\hrulefill}\\ Pembimbing Utama
\end{minipage} \hspace{0.5cm}
\begin{minipage}[b]{0.45\linewidth}
% \centering Bandung, \makebox[0.5cm]{\hrulefill}/\makebox[0.5cm]{\hrulefill}/2013\\
\vspace{2cm} Nama: \makebox[3cm]{\hrulefill}\\ Pembimbing Pendamping
\end{minipage}
\vspace{0.5cm}
}{
% \centering Bandung, \makebox[0.5cm]{\hrulefill}/\makebox[0.5cm]{\hrulefill}/2013\\
\vspace{2cm} Nama: \makebox[3cm]{\hrulefill}\\ Pembimbing Tunggal
}

\end{document}

