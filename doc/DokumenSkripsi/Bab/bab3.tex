\chapter{Analisis}
\label{chap:analisis}

\section{Mesin Navigasi KIRI}
\label{sec:mesin_navigasi_kiri}

\subsection{Mekanisme Penarikan}

Untuk mendukung integrasi data antara angkot.web.id dan KIRI, diperlukan adanya penarikan secara berkala terhadap angkot.web.id oleh KIRI. Ada dua alternatif metode yang dipertimbangkan sebagai mekanisme penarikan data ini:

\begin{description}
	\item[Metode \textit{realtime}] Metode ini mendeteksi setiap perubahan yang terjadi pada data angkot.web.id, dan langsung memberi notifikasi kepada KIRI untuk mengambil ulang data yang berubah dari angkot.web.id
	\item[Metode \textit{polling}] Pada metode ini, KIRI akan mengambil data secara berkala dari angkot.web.id, sehingga akan terdapat jeda antara perubahan data dan terbaharuinya data KIRI.
\end{description}

Dari kedua metode tersebut, peneliti memilih untuk menggunakan metode \textit{polling} dengan alasan-alasan sebagai berikut:

\begin{description}
	\item[Lebih sedikit perubahan] Angkot.web.id menggunakan protokol HTTP yang bersifat \textit{conectionless}, sehingga secara alami akan menunggu perintah dari \textit{client}, alih-alih secara aktif memberi notifikasi kepada \textit{client}. Jika menggunakan metode \textit{polling}, KIRI dapat memanfaatkan protokol \textit{Transportation Detail} yang sudah dimiliki oleh angkot.web.id.
	\item[Mengurangi kebutuhan prosesor] Rute pada KIRI dimodelkan dalam bentuk graf, dan sebelum dapat digunakan untuk pencarian graf ini harus dikonstruksi terlebih dahulu. Akibat besarnya data yang digunakan, konstruksi ini akan memakan waktu (kurang lebih 30 detik). Dengan menggunakan metode \textit{realtime}, setiap perubahan data pada kiri.web.id akan mengakibatkan graf ini harus direkonstruksi ulang. Dengan menggunakan metode \textit{polling}, penarikan data dan rekonstruksi graf dapat diatur sedemikian sehingga dilakukan pada saat pengguna sedang tidak aktif.
	\item[Urgensi Perbaikan Data] Peneliti berpendapat bahwa urgensi perbaikan data tidak terlalu tinggi untuk KIRI (bandingkan dengan perubahan harga saham, \textit{realtime GPS monitoring}, dll).
\end{description}

Penulis menetapkan penggunaan metode \textit{polling} selama 24 jam sekali, dan dipilih waktu 0.30 (setengah jam setelah tengah malam) untuk melakukan \textit{polling}. Penetapan waktu ini mempertimbangkan sedikitnya penggunaan KIRI pada saat tengah malam. Jeda setengah jam dilakukan untuk memberikan toleransi terhadap perbedaan waktu komputer beberapa detik, yang mungkin memberikan masalah pada tanggal sistem. 

\subsection{Penyimpanan Data}

Penyimpanan data diusahakan untuk meminimalisir perubahan serta menjaga kompatibilitas dengan versi sebelumnya. Seperti yang telah dibahas pada bab 2, mesin navigasi KIRI menggunakan berkas \verb/tracks.conf/ sebagai jembatan dari basis data menuju mesin navigasi. Berkas inilah yang akan dikonstruksi menjadi graf oleh kelas \verb/Worker/. Peneliti memutuskan untuk tidak mengubah format dari berkas ini, melainkan melakukan modifikasi pada basis data.

Pada basis data, tetap diusahakan untuk meminimalisir perubahan. Dari struktur tabel \verb/tracks/ yang digunakan, diidentifikasi bahwa kolom `internalInfo` tidak digunakan dalam perhitungan (tidak berpengaruh pada konstruksi graf). Oleh sebab itu, kolom ini menjadi kandidat untuk disisipkan informasi terkait dengan penarikan data dari angkot.web.id. Adapun informasi yang harus disimpan adalah:

\begin{itemize}
	\item Penanda bahwa rute ini ditarik dari angkot.web.id
	\item Kode yang mengacu pada basis data angkot.web.id, dalam hal ini \verb/id/.
\end{itemize}

Peneliti memutuskan untuk menyimpan kedua informasi tersebut dalam kolom \verb/internalInfo/, dengan format "angkotwebid:\textit{id}". Dengan cara ini, trayek yang memiliki \verb/internalInfo/ diawali dengan "angkotwebid:" akan diproses lebih lanjut dengan menarik data dari angkot.web.id. Penarikan data dilakukan dengan mendapatkan \textit{id} yang berada setelah simbol ":", dan menariknya di angkot.web.id dengan perintah \textit{Transportation Detail}. Sebagai contoh, jika ditemukan data dengan \verb/internalInfo/ bernilai "angkotwebid:1", akan ditarik data rute dari alamat \url{https://angkot.web.id/route/transportation/1.json}. Rute yang didapat tersebut kemudian akan dimasukkan ke dalam basis data pada kolom \verb/geodata/.

\section{Angkot.web.id}

\section{Optimasi}