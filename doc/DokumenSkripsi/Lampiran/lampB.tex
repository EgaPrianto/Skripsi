\chapter{Kode Program Aplikasi \textit{Game}}
\label{app:B}

%selalu gunakan single spacing untuk source code !!!!!
\singlespacing 
% language: bahasa dari kode program
% terdapat beberapa pilihan : Java, C, C++, PHP, Matlab, R, dll
%
% basicstyle : ukuran font untuk kode program
% terdapat beberapa pilihan : tiny, scriptsize, footnotesize, dll
%
% caption : nama yang akan ditampilkan di dokumen akhir, lihat contoh

\lstinputlisting[language=java,basicstyle=\tiny,caption=GyroscopeChecker.cs]{Scripts/GyroscopeChecker.cs}

\lstinputlisting[language=java,basicstyle=\tiny,caption=GameplayStatistic.cs]{Scripts/GameplayStatistic.cs}

\lstinputlisting[language=java,basicstyle=\tiny,caption=GameOverController.cs]{Scripts/Controllers/GameOverController.cs}

\lstinputlisting[language=java,basicstyle=\tiny,caption=QuestionMessageController.cs]{Scripts/Controllers/QuestionMessageController.cs}

\lstinputlisting[language=java,basicstyle=\tiny,caption=Axis.cs]{Scripts/HeadMotionDetector/Axis.cs}

\lstinputlisting[language=java,basicstyle=\tiny,caption=Detector.cs]{Scripts/HeadMotionDetector/Detector.cs}

\lstinputlisting[language=java,basicstyle=\tiny,caption=HeadMotionListener.cs]{Scripts/HeadMotionDetector/HeadMotionListener.cs}

\lstinputlisting[language=java,basicstyle=\tiny,caption=Motion.cs]{Scripts/HeadMotionDetector/Motion.cs}

\lstinputlisting[language=java,basicstyle=\tiny,caption=MotionRecorder.cs]{Scripts/HeadMotionDetector/MotionRecorder.cs}

\lstinputlisting[language=java,basicstyle=\tiny,caption=NodDetector.cs]{Scripts/HeadMotionDetector/NodDetector.cs}

\lstinputlisting[language=java,basicstyle=\tiny,caption=Pulse.cs]{Scripts/HeadMotionDetector/Pulse.cs}

\lstinputlisting[language=java,basicstyle=\tiny,caption=PulseFactory.cs]{Scripts/HeadMotionDetector/PulseFactory.cs}

\lstinputlisting[language=java,basicstyle=\tiny,caption=PulseType.cs]{Scripts/HeadMotionDetector/PulseType.cs}

\lstinputlisting[language=java,basicstyle=\tiny,caption=ShookDetector.cs]{Scripts/HeadMotionDetector/ShookDetector.cs}
